\chapter{lemma 2.3}\label{ch_lemma2_3}

\section{Change in difficulty}
In the proof of theorem 2.2 we really quickly state that:
$$E[N]^{EQUI} \leq E[N]^P,$$
intuitively this is rather clear, since one of the departurerates being higher clearly means the
queue goes quicker at some point and thus all else being equal we would expect less people to be waiting
in the queue. However this is really cumbersome to actually show via invariant distributions.
Since we now need to prove that a higher throughput at one node needs more input from lower nodes
and thus we need to scale the lower portion somewhat.

Specifically we can aim to show the following:
if $k$ is the index, whose departure rate increased and $\lambda$ is the old situation and $\mu$
is the new situation then the following should hold for all $i > k$:
$$\frac{\mu_i}{\sum_{j=k}^\infty \mu_j} = \frac{\lambda_i}{\sum_{j=k}^\infty \lambda_j}$$

We can do a similar trick with $i < k - 1$, and then we can conclude that it is only rescaling on two
sides and can prove that $\mu_k \leq \lambda_k$.

% Even more specifically we can do the following:
% By the fact that the inequalities stay the same for $i < k - 1$ we know that $\mu_i = d\cdot \lambda_i$
% furthermore we know that $\mu_{k+1}$ determines the rest of the inequalities, thus we can also scale all
% those values by $\mu_{k+1}/\lambda_{k+1}$ and thus we only need to look at determing $\mu_{k-1}$ and $\mu_{k}$
% getting a system of equations out of it that describe that $\mu_{k-1} \leq \lambda_{k-1}$ and then we are done!

\section{Step-by-step idea}

\subsection{General expression of invariant distribution}



We will first describe the invariant distribution in terms of $\lambda_0$ if there are no rates which equal zero:
\begin{lemma}\label{lem:InvariantDistributionValue}\uses{def:SchedulePolicy}
  The invariant distribution of a scheduling policy whose rate is never $0$ is of the form:
  $$\left(\prod_{i=1}^{n} \frac{\Lambda}{a_i}\right)\lambda_0$$
  at index $n$
\end{lemma}

\begin{proof}\uses{def:InvariantDistribution}
  This comes down to just using the detailed balance property of the invariant distribution and using induction
  specifically:
  \textbf{Base case}: $n = 0$ clearly $\lambda_0 = \lambda_0$.
  \textbf{Induction case}:
  The induction hypothesis is that $$\left(\prod_{i=1}^{n-1} \frac{\Lambda}{a_i}\right)\lambda_0$$
  We now need a case distribution again:
  $n = 1$ or $n \neq 1$.
  If $n = 1$ then:
    $$\Lambda \lambda_0 = a_1 \lambda_1 \implies \lambda_1 = \frac{\Lambda}{a_1} \lambda_0$$ q.e.d.
  Otherwise:
  \begin{align*}
    (\Lambda + a_{n-1})\lambda_{n-1} &= a_n \lambda_n + \Lambda \lambda_{n-2}\\
    \implies \lambda_n &= \frac{(\Lambda + a_{n-1})\lambda_{n-1} - \Lambda \lambda_{n-2}}{a_n}\\
    &\overset{IH}{=} \frac{(\Lambda + a_{n-1})\left(\prod_{i=1}^{n-1} \frac{\Lambda}{a_i}\right)\lambda_0 -
    \Lambda \left(\prod_{i=1}^{n-2} \frac{\Lambda}{a_i}\right)\lambda_0}{a_n}\\
    &= \left({(\frac{\Lambda}{a_n} + \frac{a_{n-1}}{a_n})\left(\prod_{i=1}^{n-1} \frac{\Lambda}{a_i}\right) -
    \frac{\Lambda}{a_n} \left(\prod_{i=1}^{n-2} \frac{\Lambda}{a_i}\right)}\right)\lambda_0\\
    &= \left({(\frac{\Lambda}{a_n} + \frac{a_{n-1}}{a_n})\Lambda^{n-1}\left(\prod_{i=1}^{n-1} \frac{1}{a_i}\right) -
    \frac{\Lambda}{a_n} \Lambda^{n-2 }\left(\prod_{i=1}^{n-2} \frac{1}{a_i}\right)}\right)\lambda_0\\
    &= \left({\left(\prod_{i=1}^{n} \frac{\Lambda}{a_i}\right) + \frac{a_{n-1}}{a_n}\Lambda^{n-1}\left(\prod_{i=1}^{n-1} \frac{1}{a_i}\right) -
    \frac{\Lambda}{a_n} \Lambda^{n-2 }\left(\prod_{i=1}^{n-2} \frac{1}{a_i}\right)}\right)\lambda_0\\
    &= \left({\left(\prod_{i=1}^{n} \frac{\Lambda}{a_i}\right) + \frac{a_{n-1}}{a_n} \frac{1}{a_{n-1}}\Lambda^{n-1}\left(\prod_{i=1}^{n-2} \frac{1}{a_i}\right) -
    \frac{1}{a_n} \Lambda^{n-1}\left(\prod_{i=1}^{n-2} \frac{1}{a_i}\right)}\right)\lambda_0\\
    &= \left({\left(\prod_{i=1}^{n} \frac{\Lambda}{a_i}\right) + \frac{1}{a_n}\Lambda^{n-1}\left(\prod_{i=1}^{n-2} \frac{1}{a_i}\right) -
    \frac{1}{a_n} \Lambda^{n-1}\left(\prod_{i=1}^{n-2} \frac{1}{a_i}\right)}\right)\lambda_0\\
    &= \left(\prod_{i=1}^{n} \frac{\Lambda}{a_i}\right)\lambda_0\\
  \end{align*}
  q.e.d.
\end{proof}

Then we can describe how it is normalised.
\begin{lemma}\label{lem:NormalInvariantDistributionForm}
  The invariant distribution of a scheduling policy whose rate is never $0$ is of the form:
  $$\frac{\prod_{i=1}^{n} \frac{\Lambda}{a_i}}{\sum_{n=0}^\infty \prod_{i=1}^{n} \frac{\Lambda}{a_i}}$$
  at index $n$.
\end{lemma}

\begin{proof}\uses{lem:InvariantDistributionValue}
  First we will use the previous theorem to get a general formula of the form:
  $$\prod_{i=1}^n \frac{\Lambda}{a_i} \lambda_0$$
  And then we will prove that only this distribution will sum to $1$, e.g. by for example stating
  if $$\lambda_0 > \frac{1}{\sum_{n=0}^\infty \prod_{i=1}^{n} \frac{\Lambda}{a_i}}$$
  then it sums to higher than $1$ and if:
  $$\lambda < \frac{1}{\sum_{n=0}^\infty \prod_{i=1}^{n} \frac{\Lambda}{a_i}}$$
  than it will be lower than one. This will simply be by using linearity of limits.
\end{proof}
% First we derive the invariant distribution described as the value of the
% invariant distribution at index $0$:
% \begin{lemma}\label{lem:InvariantDistributionValue}\uses{def:SchedulePolicy, def:InvariantDistribution}
%   The value of the invariant distribution $\lambda$ at index $n$ expressed as a scaling factor times the
%   invariant distribution $\lambda$ at index $i$. Given that all departure rates up to and including
%   $n$ are non-zero equals:
%   $$\lambda_i = \frac{\Lambda^{i}}{\prod_{i=1}^n b_i} \lambda_0$$
% \end{lemma}

% \begin{proof}
%   This will be a proof by induction using the invariant distribution definition.
% \end{proof}

\subsection{Non-negative distribution values}
We first will need to show that if there are unreachable states in the queue that the invariant
distribution is zero there, or more concretely:
\begin{lemma}\label{lem:UnreachableInvariantDistribution}
  If there exists unique $n \in \mathbb{N}$ such that the departure rate is $0$ at state $n$ then $\forall
  i < n$ the invariant distribution has value $0$ for state $i$.
\end{lemma}

\begin{proof}\uses{lem:InvariantDistributionValue}
  This follows rather quickly from \ref{lem:InvariantDistributionValue}. Simply state the invariant distribution
  definition at index $n-1$ and then simple rewriting yields an equality of the form:
  $$\frac{a}{b+c} = \frac{a}{b}$$
  Where we already know that $c\neq0$ and thus $a=0$ is the only correct option.
\end{proof}

% Now we also need to show that if there is a zero-valued index in our distribution that then there
% exists a value with a zero-valued departurerate.
% \begin{lemma}\label{lem:zeroValuedInvariantDistribution}
%   Suppose there exists an $n$ in our state space such that $\lambda_n = 0$ then there exists a $k > n$
%   such that $a_k = 0$.
% \end{lemma}

% \begin{proof}
%   First we will show that $\forall i < n$ $\lambda_i = 0$:
%   We will do a case distinction: Either there exists an $l < n$ such that $a_l = 0$ then by the previous lemma
%   we know that $\lambda_{l-1} = 0$ Otherwise we note that again we know that:

%   $$(a+b_{n-1})\lambda_{n-1} = a\lambda_{n-2} + b_{n}\lambda_{n}$$
%   and therefore:
%   $$(a+b_{n-1})\lambda_{n-1} = a\lambda_{n-2}$$
%   and thus $$\lambda_{n-2} = \frac{(a+b_{n-1})}{a}\lambda_{n-1}$$
%   Afterwards we can inductively proof what we want to prove:
%   $$$$
% \end{proof}

% Now we also need to show that when we get a value to be $0$ in the invariant distribution that the same recursive definition
% holds even if it now starts at $n$.

% Now we first need that if we get a zero-valued index, that then also all previous indices are $0$,
% assuming that this
% \begin{lemma}\label{lem:StartIndexWhenZero}\uses{def:InvariantDistribution}
%   Suppose there is an index $i \in \mathbb{N}$ s.t. our invariant distribution is $0$ at $i$, and
%   suppose our queue is stable, suppose there is no index with $n \leq i$ s.t. the rate is $0$.
%   Then there exists an $n$ s.t. the rate at index $n$ is $0$.
% \end{lemma}

Now we also need a lemma which will allow for shifting the general expression of the invariant distribution.
\begin{lemma}\label{lem:ShiftedInvariantDistributionValue}\uses{def:InvariantDistribution}
  Suppose we have an index $n$ s.t. we know that $\lambda_i = 0$ for all $i < n$, and furthermore
  that the departurerate at index $n$ is $0$. And we also know that $\exists m > n \in \mathbb{N}$ s.t
  $\forall k \in \mathbb{N}$ with $m \geq k > n$
  the departurerates are greater than $0$ then $\lambda_k$ is of the form:
  $$\prod_{j={n+1}}^k \frac{\Lambda}{a_j} \lambda_n$$
\end{lemma}

\begin{proof}
  This will just be copying the proof from \ref{lem:InvariantDistributionValue}, but this time instead
  of using the property that at index $0$ we have that our markov chain only goes in one direction,
  we now plug-in $0$ our zero value of $\lambda_{n-1}$ and $a_n = 0$ (as the departurerate is $0$ at this index).
\end{proof}

% Then we get the highest such values to allow for nice induction.
\begin{lemma}\label{lem:MaxUnreachableInvariance}\uses{lem:UnreachableInvariantDistribution}
  If there exists finite $A \subsetneq \mathbb{N}$ such that for all $n \in A$ the departure rate at
  state $n$ is zero and there exists no $n \notin A$ with departurerate $0$.
  Then the invariant distribution has value $0$ at $i < \max(A)$.
\end{lemma}
\begin{proof}
  We will need to do induction over $i$.
  % Specifically we will need to show that if $\min_{n \in A}\{i \leq n\}$ then
  % the value at $i$ is $0$.
  % i.e.
  base case $i = 0$:\\
  \qquad then use \ref{lem:UnreachableInvariantDistribution}.
  induction case $i$:\\
  \qquad we know that the invariant distribution of $i-1$ is zero. If the departurerate of $i$ is not
  zero i.e. $i \notin A$ then simply apply induction hypothesis. Otherwise apply previous lemma and note that
  if we let $m = \min{n \in A \colon n > i} - 1$ then we have two cases:
  \begin{enumerate}
  \item if $m = i$ then we have:
  $$(\Lambda+a_i)\lambda_i = \Lambda\lambda_{i-1} + a_{i+1} \lambda_{i+1}$$
  Using the definitions we find that the RHS is $0$ and since $\Lambda > 0, a_i \geq 0$ we find that
  $\lambda_i = 0$.
  \item otherwise we instead find the following set of equations:
  $$\lambda_m = \prod_{j=i+1}^m \frac{\Lambda}{a_j} \lambda_i,$$
  from the previous lemma. But also:
  $$(\Lambda+a_m)\lambda_m = \Lambda \lambda_{m-1} + a_{m+1}\lambda_{m+1}$$
  and plugging in $a_{m+1} = 0$ we get:
  $$\lambda_m = \frac{\Lambda}{\Lambda + a_m}\lambda_{m-1}$$
  Which can again be rewritten using the formula from the previous lemma to:
  $$\frac{\Lambda}{\Lambda + a_m}\prod_{j=i+1}^{m-1} \frac{\Lambda}{a_j}$$
  Now we can then substitute $\lambda_m$ in the previous equation to get:
  \begin{align*}
  \prod_{j=i+1}^m \frac{\Lambda}{a_j} \lambda_i &= \frac{\Lambda}{\Lambda + a_m}\prod_{j=i+1}^{m-1} \frac{\Lambda}{a_j}\\
  \frac{\Lambda}{a_m}\prod_{j=i+1}^{m-1} \frac{\Lambda}{a_j} \lambda_i &= \frac{\Lambda}{\Lambda + a_m}\prod_{j=i+1}^{m-1} \frac{\Lambda}{a_j}\\
  (\frac{\Lambda}{a_m} - \frac{\Lambda}{\Lambda + a_m})\prod_{j=i+1}^{m-1} \frac{\Lambda}{a_j} \lambda_i &= 0\\
  (\frac{\Lambda}{a_m} - \frac{\Lambda}{\Lambda + a_m}) = 0 &\vee \prod_{j=i+1}^{m-1} \frac{\Lambda}{a_j} \lambda_i = 0\\
  (\frac{\Lambda(\Lambda + a_m)-\Lambda^2}{a_m(\Lambda + a_m)}) = 0 &\vee \lambda_i = 0\\
  \Lambda(\Lambda + a_m)-\Lambda^2 = 0 &\vee \lambda_i = 0\\
  \Lambda(\Lambda + a_m)-\Lambda^2 = 0 &\vee \lambda_i = 0\\
  \Lambda a_m = 0 &\vee \lambda_i = 0\\
  \text{contradiction!} &\vee \lambda_i = 0\\
  \end{align*}
  Thus $\lambda_i = 0$.
\end{enumerate}
\end{proof}


\subsection{Calculating the changed invariant distribution}

\begin{lemma}\label{lem:BeforeCutScaled}\uses{lem:InvariantDistributionValue, lem:NormalInvariantDistributionForm}
  For any two scheduling policies whose departure rates differ at exactly one index $n$, $\forall i \in \mathbb{N}$
  with $i \leq n - 1$ the invariant distribution at index $i$ differs only a constant $c \in \mathbb{R}$ or $c = \infty$.
\end{lemma}
\begin{proof}
  We will need to do a case distinction: either the rate is $0$ or it is isn't.\\
  If it is: Apply previous lemma.\\
  If it isn't: Simply apply \ref{lem:NormalInvariantDistributionForm}
\end{proof}

\begin{lemma}\label{lem:AfterCutScaled}\uses{def:SchedulePolicy, lem:NormalInvariantDistributionForm}
  For any two scheduling policies whose departure rates differ at exactly one index $n$, $\forall i \in \mathbb{N}$
  with $i \geq n$ the invariant distribution differs only a constant $c \in \mathbb{R}$ or $c = \infty$.
\end{lemma}

\begin{proof}\uses{lem:ShiftedInvariantDistributionValue}
  We will need to do a case distinction: either the rate is $0$ or it is isn't.\\
  If it is: Apply the shifted invariant distribution's value lemma.\\
  If it isn't: Simply apply \ref{lem:NormalInvariantDistributionForm}
\end{proof}

Now we also need to find in which way our $c$ constants have changed. Luckily there is only
one logical value to choose here.
\begin{lemma}\label{lem:InvariantDistributionsSmaller}\uses{lem:MaxUnreachableInvariance,
  lem:ShiftedInvariantDistributionValue, lem:BeforeCutScaled, lem:AfterCutScaled}
  For any two scheduling policies $P$ and $Q$ whose departure rates differ at one index $n$,
  and $P's$ departure rate at this index is higher than $Q$'s. Then $Q$'s invariant distribution is
  a constant $c \leq 1$ smaller than $P$ at the indices $i \leq n - 1$ and $Q$'s invariant distribution is
  a constant $C \geq 1$ bigger than $P$'s at the indices $i \geq n$.
\end{lemma}

For the proof the general idea of how the indices can be written as the product of the ratios of
the arrival rates and departure rates is very important.
\begin{proof}
  Clearly we first use \ref{lem:BeforeCutScaled, lem:AfterCutScaled}
  We will need to do a few case distinctions:
  \begin{enumerate}
    \item First of all if there exists an index $m > n$ with departure rate $0$. Then
    both $P$ and $Q$'s invariant distribution is $0$ for all indices $i < m$ by \ref{lem:MaxUnreachableInvariance}. Thus $c = 1$ in the $i < n$ case.
    Furthermore by \ref{lem:ShiftedInvariantDistributionValue} we find that both invariant distributions
    are the same and therefore $c = 1$ in both cases.
    \item Otherwise if $Q$'s departure rate is $0$ at index $n$ then by \ref{lem:UnreachableInvariantDistribution} $Q$'s invariant distribution
    for all indices $i < n$ equals $0$, whereas $P$'s departurerate is at least non-zero for $i = n-1$ by
    \ref{lem:ShiftedInvariantDistributionValue} or \ref{lem:InvariantDistributionValue} and using that
    the sum of an invariant distribution must be $1$. Thus clearly $c=0$ for all $i < n$. Furthermore
    by the fact that the invariant distribution must sum to $1$ and $P$ has non-zero values we know that
    $\sum_{i=n}^\infty \lambda_{P,i} = 1 - \sum_{i=0}^{n-1}\lambda_{P,i} < 1$. We see that, because
    $\sum_{i=n}^\infty \lambda_{P,i} < 1$ and that $\sum_{i=n}^\infty \lambda_{Q,i} = 1$.
    Now using \ref{lem:ShiftedInvariantDistributionValue} or \ref{lem:InvariantDistributionValue}
    on $P$ and \ref{lem:ShiftedInvariantDistributionValue} for $Q$ we find that $C = 1/\sum_{i=n}^\infty \lambda_{P,i}$.
    \item If $Q$'s departure rate is non-zero then we find using \ref{lem:ShiftedInvariantDistributionValue} or \ref{lem:InvariantDistributionValue}
    for both of them, that after $n$ that our product of $P$ must contain a smaller fraction compared to $\lambda_0$
    or $\lambda_k$ for some $k < n$. Specifically $P$'s fraction is $\frac{a_{P,n}}{a_{Q,n}}$ times smaller from then on.
    Since $\sum_{i=0}^\infty \lambda_{P,i} = 1$ we find that:
    \begin{align}
      1 &= \sum_{i=0}^{n-1} \lambda_{P,i} + \sum_{i=n}^{\infty} \lambda_{P,i} \nonumber\\
        &= \frac{\lambda_{P,0}}{\lambda_{Q,0}} \sum_{i=0}^{n-1} \lambda_{Q,i}  + \frac{\lambda_{P,0}}{\lambda_{Q,0}}\frac{a_{Q,n}}{a_{P,n}}\sum_{i=n}^{\infty} \lambda_{Q,i} \nonumber\\
        &= \left(\sum_{i=0}^{n-1} \lambda_{Q,i}  + \frac{a_{Q,n}}{a_{P,n}}\sum_{i=n}^{\infty} \lambda_{Q,i}\right) \frac{\lambda_{P,0}}{\lambda_{Q,0}} \label{eqn} \hspace{36pt}(1) % Sadly eqn ref is not working correctly :(
        % &= \left(\sum_{i=0}^{n-1} \lambda_{Q,i}  + \frac{a_{Q,n}}{a_{P,n}}\sum_{i=n}^{\infty} \lambda_{Q,i}\right) \frac{\lambda_{P,0}}{\lambda_{Q,0}}\\
    \end{align}
    Now note that:
    $$\left(\sum_{i=0}^{n-1} \lambda_{Q,i}  + \frac{a_{Q,n}}{a_{P,n}}\sum_{i=n}^{\infty} \lambda_{Q,i}\right) < 1$$
    by the fact that
    $$\sum_{i=0}^{n-1} \lambda_{Q,i}  + \sum_{i=n}^{\infty} \lambda_{Q,i} = 1$$
    and
    $$\frac{a_{Q,n}}{a_{P,n}} < 1$$
    (thus using concavity of summation :D).
    Thus $\frac{\lambda_{P,0}}{\lambda_{Q,0}} > 1 \implies \lambda_{P,0} > \lambda_{Q,0}$, and therefore
    $c < 1$ for $i \leq n - 1$ and $C > 1$ for $i \geq n$, (because it must still sum to $1$).
    Furthermore $c$ is of the form:
    $$1 + (\frac{a_{Q,n}}{a_{P,n}}-1)\sum_{i=n}^{\infty}\lambda_{Q,i}$$
    by:
    \begin{align*}
      \frac{\lambda_{P,0}}{\lambda_{Q,0}} &= \frac{1}{\sum_{i=0}^{n-1}\lambda_{Q,i} + \frac{a_{Q,n}}{a_{P,n}} \sum_{i=n}^\infty \lambda_{Q,i}}\\
      \iff \lambda_{Q,0} &= \left(\sum_{i=0}^{n-1}\lambda_{Q,i} + \frac{a_{Q,n}}{a_{P,n}} \sum_{i=n}^\infty \lambda_{Q,i}\right)\lambda_{P,0}\\
      &=\left(1-\sum_{i=n}^{\infty}\lambda_{Q,i} + \frac{a_{Q,n}}{a_{P,n}} \sum_{i=n}^\infty \lambda_{Q,i}\right)\lambda_{P,0}\\
      &=\left(1 + (\frac{a_{Q,n}}{a_{P,n}}-1) \sum_{i=n}^\infty \lambda_{Q,i}\right)\lambda_{P,0}\\
    \end{align*}
    And $C$ is of the form:
    $$C = \frac{a_{P,n}}{a_{Q,n}}c = 1 + (\frac{a_{P,n}}{a_{Q,n}}-1) \sum_{i=0}^{n-1}\lambda_{Q,i}$$
    this follows from:
    $$\sum_{i=n}^\infty\lambda_{Q,i} = \frac{a_{P,n}}{a_{Q,n}}\frac{\lambda_{Q,0}}{\lambda_{P,0}}\sum_{i=n}^\infty{\lambda_{P,i}}$$
    again by \ref{lem:ShiftedInvariantDistributionValue} or \ref{lem:InvariantDistributionValue}.
    Plugging in $\lambda_{Q,0} = c \lambda_{P,0}$ yields:
    $$\frac{a_{P,n}}{a_{Q,n}}\frac{c\lambda_{P,0}}{\lambda_{P,0}}\sum_{i=n}^\infty{\lambda_{P,i}}$$
    And therefore we clearly attain:
    \begin{align*}
      C &= \frac{a_{P,n}}{a_{Q,n}} c \\
        &= \frac{a_{P,n}}{a_{Q,n}}\left(1 + (\frac{a_{Q,0}}{a_{P,0}}-1)\sum_{i=n}^{\infty}\lambda_{Q,i}\right)\\
        &= \frac{a_{P,n}}{a_{Q,n}}\left(1 + (\frac{a_{Q,0}}{a_{P,0}}-1)(1-\sum_{i=0}^{n-1}\lambda_{Q,i})\right)\\
        &= \frac{a_{P,n}}{a_{Q,n}} + (1-\frac{a_{P,n}}{a_{Q,n}})(1-\sum_{i=0}^{n-1}\lambda_{Q,i})\\
        &= \frac{a_{P,n}}{a_{Q,n}} + 1 - \frac{a_{P,n}}{a_{Q,n}} - (1-\frac{a_{P,n}}{a_{Q,n}})\sum_{i=0}^{n-1}\lambda_{Q,i}\\
        &= 1 - (1-\frac{a_{P,n}}{a_{Q,n}})\sum_{i=0}^{n-1}\lambda_{Q,i}\\
        &= 1 + (\frac{a_{P,n}}{a_{Q,n}}-1)\sum_{i=0}^{n-1}\lambda_{Q,i}
    \end{align*}
    % Now because we have directly computed: $$\frac{\lambda_{P,0}}{\lambda_{Q,0}} = \frac{1}{c}$$
    % We can conclude using \eqref{eqn} that $C = \frac{a_{Q,n}}{a_{P,n}} c$
    % And all $i$ must be by \ref{lem:ShiftedInvariantDistributionValue} or \ref{lem:InvariantDistributionValue} writen as:
    % $$\lambda_{\_, k} = \prod_{i=m+1}^k \frac{\Lambda}{a_{\_,i}} \lambda_{m}$$
    % For some $m < n$ and therefore:
    % $$C = \frac{a_{}}{}$$
  \end{enumerate}
\end{proof}

\begin{corollary}\label{cor:NumberInTheSystemSmaller}\uses{lem:InvariantDistributionsSmaller}
  In the previous situation $\mathbb{E}[N]^P \leq \mathbb{E}[N]^Q$.
\end{corollary}

\begin{proof}
  Because $c \leq 1$ for $i < n$ and $C \geq 1$ for $i \geq n$ we find that:
  $$
   \sum_{i=0}^{n-1} i \lambda_{Q,i} + \sum_{i=n}^\infty i \lambda_{Q,i} =
   c \cdot \sum_{i=0}^{n-1} i \lambda_{P,i} + C \cdot \sum_{i=n}^\infty i \lambda_{P,i}
  $$
  We will use the same case distinctions again:
  \begin{enumerate}
   \item Clearly if $c = C = 1$ then it is true;
   \item If $Q$'s rate at index $n$ is $0$ then we have that the $c$ term yields $0$
   and because $C = \frac{1}{\sum_{i=n}^{\infty}\lambda_{P,i}}$, we find that:
   \begin{align*}
     C \cdot \sum_{i=n}^\infty i \lambda_{P,i} &= \frac{1}{\sum_{i=n}^{\infty}\lambda_{P,i}}\sum_{i=n}^\infty i \lambda_{P,i}\\
     &= (\frac{1}{\sum_{i=n}^{\infty}\lambda_{P,i}} - 1 + 1)\sum_{i=n}^\infty i \lambda_{P,i}\\
     &= (\frac{1}{\sum_{i=n}^{\infty}\lambda_{P,i}} - 1)\sum_{i=n}^\infty i \lambda_{P,i} + \sum_{i=n}^\infty i \lambda_{P,i}\\
     &\geq n(\frac{1}{\sum_{i=n}^{\infty}\lambda_{P,i}} - 1)\sum_{i=n}^\infty \lambda_{P,i} + \sum_{i=n}^\infty i \lambda_{P,i}\\
     &= n(1 - \sum_{i=n}^\infty \lambda_{P,i}) + \sum_{i=n}^\infty i \lambda_{P,i}\\
     &= n(1 - (1-\sum_{i=0}^{n-1} \lambda_{P,i})) + \sum_{i=n}^\infty i \lambda_{P,i}\\
     &= n\sum_{i=0}^{n-1} \lambda_{P,i} + \sum_{i=n}^\infty i \lambda_{P,i}\\
     &\geq \sum_{i=0}^{n-1}i \lambda_{P,i} + \sum_{i=n}^\infty i \lambda_{P,i}
    %  &= \frac{1-\sum_{i=0}^{n-1}\lambda_{P,i}}{\sum_{i=0}^{n-1}\lambda_{P,i}}\sum_{i=n}^\infty i \lambda_{P,i}  + \sum_{i=n}^\infty i \lambda_{P,i}\\
    % %  &\geq n\frac{1-\sum_{i=0}^{n-1}\lambda_{P,i}}{\sum_{i=0}^{n-1}\lambda_{P,i}}\sum_{i=n}^\infty \lambda_{P,i}  + \sum_{i=n}^\infty i \lambda_{P,i}\\
    %  &= \frac{\sum_{i=n}^{\infty}\lambda_{P,i}}{\sum_{i=0}^{n-1}\lambda_{P,i}}\sum_{i=n}^\infty i\lambda_{P,i}  + \sum_{i=n}^\infty i \lambda_{P,i}\\
   \end{align*}
   \item If both $Q$'s rate at index $n$ is not $0$ then we have:
   \begin{align*}
    &c \cdot \sum_{i=0}^{n-1} i \lambda_{P,i} + C \cdot \sum_{i=n}^\infty i \lambda_{P,i} \\
    &= \left(1 + (\frac{a_{Q,n}}{a_{P,n}}-1)\sum_{i=n}^{\infty}\lambda_{Q,i}\right)  \sum_{i=0}^{n-1} i \lambda_{P,i} + \left(1 + (\frac{a_{P,n}}{a_{Q,n}}-1) \sum_{i=0}^{n-1}\lambda_{Q,i}\right) \sum_{i=n}^\infty i \lambda_{P,i}
    % &\geq \sum_{i=0}^{n-1} i \lambda_{P,i} + \sum_{i=n}^\infty i \lambda_{P,i} \iff
   \end{align*}
   It is evident that this expression is greater than or equal to $$\sum_{i=0}^{n-1}i \lambda_{P,i} + \sum_{i=n}^\infty i \lambda_{P,i}$$
   if and only if:
   $$(\frac{a_{Q,n}}{a_{P,n}}-1)\left(\sum_{i=n}^{\infty}\lambda_{Q,i}\right)\left(\sum_{i=0}^{n-1} i \lambda_{P,i}\right) + (\frac{a_{P,n}}{a_{Q,n}}-1) \left(\sum_{i=0}^{n-1}\lambda_{Q,i}\right)\left(\sum_{i=n}^\infty i \lambda_{P,i}\right)$$
   Thus we will prove this instead:
   \begin{align*}
    &(\frac{a_{Q,n}}{a_{P,n}}-1)\left(\sum_{i=n}^{\infty}\lambda_{Q,i}\right)\left(\sum_{i=0}^{n-1} i \lambda_{P,i}\right) + (\frac{a_{P,n}}{a_{Q,n}}-1) \left(\sum_{i=0}^{n-1}\lambda_{Q,i}\right)\left(\sum_{i=n}^\infty i \lambda_{P,i}\right) \\
    &= (\frac{a_{Q,n} - a_{P,n}}{a_{P,n}})\left(\sum_{i=n}^{\infty}\lambda_{Q,i}\right)\left(\sum_{i=0}^{n-1} i \lambda_{P,i}\right) + (\frac{a_{P,n} - a_{Q,n}}{a_{Q,n}}) \left(\sum_{i=0}^{n-1}\lambda_{Q,i}\right)\left(\sum_{i=n}^\infty i \lambda_{P,i}\right) \\
    &= (\frac{a_{Q,n}^2 - a_{P,n}a_{Q,n}}{a_{P,n}a_{Q,n}})\left(\sum_{i=n}^{\infty}\lambda_{Q,i}\right)\left(\sum_{i=0}^{n-1} i \lambda_{P,i}\right) + (\frac{a_{P,n}^2 - a_{P,n}a_{Q,n}}{a_{P,n}a_{Q,n}}) \left(\sum_{i=0}^{n-1}\lambda_{Q,i}\right)\left(\sum_{i=n}^\infty i \lambda_{P,i}\right)
   \end{align*}
   Again this is only greater than or equal to $0$ if $a_{P,n}a_{Q,n}$ times the expression is zero,
   thus we can prove the same for the simplified expression:
   \begin{align*}
    &(a_{Q,n}^2 - a_{P,n}a_{Q,n})\left(\sum_{i=n}^{\infty}\lambda_{Q,i}\right)\left(\sum_{i=0}^{n-1} i \lambda_{P,i}\right) + (a_{P,n}^2 - a_{P,n}a_{Q,n}) \left(\sum_{i=0}^{n-1}\lambda_{Q,i}\right)\left(\sum_{i=n}^\infty i \lambda_{P,i}\right)\\
    &\geq (a_{Q,n}^2 - a_{P,n}a_{Q,n})\left(C\sum_{i=n}^{\infty}\lambda_{P,i}\right)\left(\sum_{i=0}^{n-1} i \lambda_{P,i}\right) + (a_{P,n}^2 - a_{P,n}a_{Q,n}) \left(\sum_{i=0}^{n-1}\lambda_{Q,i}\right)\left(\sum_{i=n}^\infty i \lambda_{P,i}\right)\\
    &= (a_{Q,n}^2 - a_{P,n}a_{Q,n})\left(\frac{a_{P,n}}{a_{Q,n}}c\sum_{i=n}^{\infty}\lambda_{P,i}\right)\left(\sum_{i=0}^{n-1} i \lambda_{P,i}\right) + (a_{P,n}^2 - a_{P,n}a_{Q,n}) \left(\sum_{i=0}^{n-1}\lambda_{Q,i}\right)\left(\sum_{i=n}^\infty i \lambda_{P,i}\right)\\
    &= (a_{P,n}a_{Q,n} - a_{P,n}^2)\left(\sum_{i=n}^{\infty}\lambda_{P,i}\right)\left(c\sum_{i=0}^{n-1} i \lambda_{P,i}\right) + (a_{P,n}^2 - a_{P,n}a_{Q,n}) \left(\sum_{i=0}^{n-1}\lambda_{Q,i}\right)\left(\sum_{i=n}^\infty i \lambda_{P,i}\right)\\
    &= -(a_{P,n}^2- a_{P,n}a_{Q,n})\left(\sum_{i=n}^{\infty}\lambda_{P,i}\right)\left(\sum_{i=0}^{n-1} i \lambda_{Q,i}\right) + (a_{P,n}^2 - a_{P,n}a_{Q,n}) \left(\sum_{i=0}^{n-1}\lambda_{Q,i}\right)\left(\sum_{i=n}^\infty i \lambda_{P,i}\right)\\
   \end{align*}
   Now notice that $-(a_{P,n}^2- a_{P,n}a_{Q,n}) \leq 0$ by $a_{P,n} \geq a_{Q,n}$ and thus we can
   ............ to $n-1$ whereas the right hand term can instead be changed to $n$ i.e. we get:
   \begin{align*}
    &-(a_{P,n}^2- a_{P,n}a_{Q,n})\left(\sum_{i=n}^{\infty}\lambda_{P,i}\right)\left(\sum_{i=0}^{n-1} i \lambda_{Q,i}\right) + (a_{P,n}^2 - a_{P,n}a_{Q,n}) \left(\sum_{i=0}^{n-1}\lambda_{Q,i}\right)\left(\sum_{i=n}^\infty i \lambda_{P,i}\right)\\
    &\geq -(a_{P,n}^2- a_{P,n}a_{Q,n})\left(\sum_{i=n}^{\infty}\lambda_{P,i}\right)\left((n-1)\sum_{i=0}^{n-1} \lambda_{Q,i}\right) + (a_{P,n}^2 - a_{P,n}a_{Q,n}) \left(\sum_{i=0}^{n-1}\lambda_{Q,i}\right)\left(n\sum_{i=n}^\infty \lambda_{P,i}\right)\\
    &= -(n-1)(a_{P,n}^2- a_{P,n}a_{Q,n})\left(\sum_{i=n}^{\infty}\lambda_{P,i}\right)\left(\sum_{i=0}^{n-1} \lambda_{Q,i}\right) + n(a_{P,n}^2 - a_{P,n}a_{Q,n}) \left(\sum_{i=0}^{n-1}\lambda_{Q,i}\right)\left(\sum_{i=n}^\infty \lambda_{P,i}\right)\\
    &= (a_{P,n}^2- a_{P,n}a_{Q,n})\left(\sum_{i=n}^{\infty}\lambda_{P,i}\right)\left(\sum_{i=0}^{n-1} \lambda_{Q,i}\right)
   \end{align*}
   Now note that $(a_{P,n}^2- a_{P,n}a_{Q,n}) \geq 0$, $\sum_{i=n}^{\infty}\lambda_{P,i} \geq 0$,
   $\sum_{i=0}^{n-1} \lambda_{Q,i} \geq 0$ and therefore likewise their product is greater than or
   equal to zero, which completes the proof.
  \end{enumerate}

\end{proof}

\subsection{Our goal}
Now we have everything we need to actually prove our theorem.
\begin{theorem}\label{thm:theorem2.3}\uses{cor:NumberInTheSystemSmaller}
  If we have two scheduling policies $P$ and $Q$. And $P's$ departurerates are always higher than
  or equal to $Q$'s and let $Q$ only have finite zero-valued departure rates. Then: $$\mathbb{E}[N]^{P} \leq \mathbb{E}[N]^{Q}.$$
\end{theorem}

\begin{proof}
  We will create a set of intermediary policies between $P$ and $Q$ to prove what we wanted to prove.
  Let $n \in \mathbb{N}$ and let $R_n$ denote the policy whose departure rate at index $j \leq n$ equals
  $P$'s departure rate and at index $j > n$ $Q$'s. Then using induction on $n$ we find that:
  $\mathbb{E}[N]^{Q} \geq \mathbb{E}[N]^{R_i}$. Furthermore using induction on $n$ we find that
  $\mathbb{E}[N]^{R_n}$ is monotonically increasing. Lastly after cleaning up the proof of \ref{cor:NumberInTheSystemSmaller}
  we actually find that our $$\mathbb{E}[N]^P = \lim_{n \to \infty} \mathbb{E}[N]^{R_i}$$
\end{proof}
% \begin{lemma}\label{lem:}
