\chapter{lemma 2.3}\label{ch_lemma2_3}

\section{Change in difficulty}
In the proof of theorem 2.2 we really quickly state that:
$$E[N]^{EQUI} \leq E[N]^P,$$
intuitively this is rather clear, since one of the departurerates being higher clearly means the
queue goes quicker at some point and thus all else being equal we would expect less people to be waiting
in the queue. However this is really cumbersome to actually show via invariant distributions.
Since we now need to prove that a higher throughput at one node needs more input from lower nodes
and thus we need to scale the lower portion somewhat.

Specifically we can aim to show the following:
if $k$ is the index, whose departure rate increased and $\lambda$ is the old situation and $\mu$
is the new situation then the following should hold for all $i > k$:
$$\frac{\mu_i}{\sum_{j=k}^\infty \mu_j} = \frac{\lambda_i}{\sum_{j=k}^\infty \lambda_j}$$

We can do a similar trick with $i < k - 1$, and then we can conclude that it is only rescaling on two
sides and can prove that $\mu_k \leq \lambda_k$.

% Even more specifically we can do the following:
% By the fact that the inequalities stay the same for $i < k - 1$ we know that $\mu_i = d\cdot \lambda_i$
% furthermore we know that $\mu_{k+1}$ determines the rest of the inequalities, thus we can also scale all
% those values by $\mu_{k+1}/\lambda_{k+1}$ and thus we only need to look at determing $\mu_{k-1}$ and $\mu_{k}$
% getting a system of equations out of it that describe that $\mu_{k-1} \leq \lambda_{k-1}$ and then we are done!

\section{Step-by-step idea}

\subsection{General expression of invariant distribution}
First we derive the invariant distribution described as the value of the
invariant distribution at index $0$:
\begin{lemma}\label{lem:InvariantDistributionValue}\uses{def:SchedulePolicy, def:InvariantDistribution}
  The value of the invariant distribution $\lambda$ at index $n$ expressed as a scaling factor times the
  invariant distribution $\lambda$ at index $i$. Given that all departure rates up to and including
  $n$ are non-zero equals:
  $$\lambda_i = \frac{\Lambda^{i}}{\prod_{i=1}^n b_i} \lambda_0$$
\end{lemma}

\begin{proof}
  This will be a proof by induction using the invariant distribution definition.
\end{proof}

\subsection{Non-negative distribution values}
We first will need to show that if there are unreachable states in the queue that the invariant
distribution is zero there, or more concretely:
\begin{lemma}\label{lem:UnreachableInvariantDistribution}\uses{def:SchedulePolicy, def:InvariantDistribution}
  If there exists unique $n \in \mathbb{N}$ such that the departure rate is $0$ at state $n$ then $\forall
  i < n$ the invariant distribution has value $0$ for state $i$.
\end{lemma}

\begin{proof}
  This follows rather quickly from \ref{lem:InvariantDistributionValue}. Simply state the invariant distribution
  definition at index $n-1$ and then simple rewriting yields an equality of the form:
  $$\frac{a}{b+c} = \frac{a}{b}$$
  Where we already know that $c\neq0$ and thus $a=0$ is the only correct option.
\end{proof}

Then we get the highest such values to allow for nice induction.
\begin{lemma}\label{lem:MaxUnreachableInvariance}\uses{def:SchedulePolicy, def:InvariantDistribution}
  If there exists finite $A \subsetneq \mathbb{N}$ such that for all $n \in A$ the departure rate at
  state $n$ is zero then the invariant distribution has value $0$ at $i < \max(A)$.
\end{lemma}



\subsection{Calculating the changed invariant distribution}2
\begin{lemma}\label{lem:BeforeCutScaled}\uses{def:SchedulePolicy, def:InvariantDistribution}
  For any two scheduling policies whose departure rates differ at exactly one index $i$, $\forall n \in \mathbb{N}$
  with $n \leq i - 2$ the invariant distribution differs only a constant $c \in \mathbb{R}$.
\end{lemma}

\begin{lemma}\label{lem:AfterCutScaled}\uses{def:SchedulePolicy, def:InvariantDistribution}
  For any two scheduling policies whose departure rates differ at exactly one index $i$, $\forall n \in \mathbb{N}$
  with $n > i $ the invariant distribution differs only a constant $c \in \mathbb{R}$.
\end{lemma}
