\chapter{lemma 2.3}\label{ch_lemma2_3}

\section{Change in difficulty}
In the proof of theorem 2.2 we really quickly state that:
$$E[N]^{EQUI} \leq E[N]^P,$$
intuitively this is rather clear, since one of the departurerates being higher clearly means the
queue goes quicker at some point and thus all else being equal: we would expect less people to be waiting
in the queue. However this is really cumbersome to actually show via invariant distributions,
since we now need to prove that a higher throughput at one node, needs more input from lower nodes
and thus we need to scale the lower portion somewhat.

Specifically we can aim to show the following:
if $k$ is the index, whose departure rate increased and $\lambda$ is the old situation and $\mu$
is the new situation then the following should hold for all $i > k$:
$$\frac{\mu_i}{\sum_{j=k}^\infty \mu_j} = \frac{\lambda_i}{\sum_{j=k}^\infty \lambda_j}$$

We can do a similar trick with $i < k - 1$, and then we can conclude that it is only rescaling on two
sides and can prove that $\mu_k \leq \lambda_k$.

Even more specifically we can do the following:
By the fact that the inequalities stay the same for $i < k - 1$ we know that $\mu_i = d\cdot \lambda_i$
furthermore we know that $\mu_{k+1}$ determines the rest of the inequalities, thus we can also scale all
those values by $\mu_{k+1}/\lambda_{k+1}$ and thus we only need to look at determing $\mu_{k-1}$ and $\mu_{k}$
getting a system of equations out of it that describe that $\mu_{k-1} \leq \lambda_{k-1}$ and then we are done!
