\chapter{definitions}\label{ch_definitions}

\section{General Idea}
Before we can start proving properties about our actual mathematical object we first need to actually
start writing down what our definitions are. Thus what we aim to do here is write down all the definitions
we require.

\section{The definitions}

\begin{definition}\label{BSC.corespace}\lean{BscThesisFormalisation.definitions.coreSpace}\leanok
   \textit{coreSpace} is defined to be the actual allowed allotment of cores i.e. if given a vector
   $s \in \mathbb{R}^n$ describing how many cores we have then our corespace becomes:
   $x \in \mathbb{R}^n$ with the requirement that: $0 \leq x_i \leq s_i$ forall $1 \leq i \leq n$.
   After this we shall abbreviate this space as $C_s$.
\end{definition}

The Mathlib library already defines concavity more generally. We redefine our own instance to make
it easier to work with:
\begin{definition}\label{BSC.myConcave}\lean{BscThesisFormalisation.definitions.myConcave}\leanok
  A function $f$ is said to be \textit{concave} on $s \in \mathbb{R^n}$ the core vector if $\forall x, y \in C_s$
  and all $a, b \in \mathbb{R}$ with the property that $a + b = 1$ implies that:
  $$a \cdot f(x) + b \cdot f(y) \leq f(a \cdot x + b \cdot y)$$
\end{definition}

\begin{definition}\label{BSC.Sublinear}\lean{BscThesisFormalisation.definitions.Sublinear}\leanok
