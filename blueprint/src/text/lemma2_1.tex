\chapter{lemma 2.1}\label{ch_lemma2_1}

\section{General Idea}
What we would like to do is mostly just copy the proof from the thesis, since almost
nothing happens in it anyway. For this we need the following definitions:
\begin{enumerate}
  \item concave
  \item sublinear
  \item speedup.
\end{enumerate}

\begin{lemma}\label{lem:lemma2.1}\uses{def:SpeedUpFunction, def:Sublinear, def:myConcave}\lean{BscThesisFormalisation.lemma2_1.lemma2_1}
  For any concave, sublinear speedup function, $s$ with $\alpha \in \mathbb{R}^n$,
  the function $i\cdot s(\frac{\alpha}{i})$ is increasing in $i$ for all $i<||\alpha||_1$,
  and is non-decreasing in $i$ for all $i\geq ||\alpha||_1$.
\end{lemma}
\begin{proof}
  We will proof the following two different cases:
  \begin{enumerate}
      \item[\textbf{Case 1}]  $i < ||\alpha||_1$: We need to prove in this case that $i\cdot s(\frac{\alpha}{i})$ is increasing. We will look at the following difference for any $\delta > 0$:
        $$i (1+\delta) s \left(\frac{\alpha}{i\cdot(1+\delta)} \right) - i \cdot s (\frac{\alpha}{i}) = i \left((1+\delta) s \left(\frac{\alpha}{i\cdot(1+\delta)} \right) - s (\frac{\alpha}{i})\right)$$
      Since $||\frac{\alpha}{i}|| > 1$, $s$ increases sublinearly, and $(1+\delta)s\left(\frac{\alpha}{i \cdot (1+\delta)}\right) > s(\frac{\alpha}{i})$. This yields
        $$i \cdot(1+\delta)s\left(\frac{\alpha}{i\cdot(1+\delta)} \right) - i \cdot s\left(\frac{\alpha}{i}\right) > 0$$
      From which we can conclude that $i \cdot s(\frac{\alpha}{i})$ is increasing in $i$.
      \item[\textbf{Case 2}] $i \geq ||\alpha||_1$: We need to prove in this case that $i\cdot s(\frac{\alpha}{i})$ is non-decreasing. This follows directly from the assumption that $s\left(\frac{\alpha}{i}\right) = \frac{\alpha \cdot \beta}{i}$. From which we conclude that $$i\cdot s \left(\frac{\alpha}{i}\right) = \alpha \cdot \beta \qquad \forall i \geq ||\alpha||_1$$
      which is non-decreasing in $i$.
  \end{enumerate}
\end{proof}
